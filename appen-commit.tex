\section{Commitment Scheme}
We show the basic protocols that implement commitment functionality in this appendix.\change{Add intros}
\subsection{Commitment Schemes}
A commitment scheme is an efficient {\it two-phase two-party protocol\/} between a sender and a receiver where the sender can commit itself to a {\it value\/} such that:
\begin{enumerate}
\item {\it Hiding\/}: The receiver does not gain any knowledge of the sender's committed {\it value\/} at the end of the first phase.
\item {\it Binding\/}: Given the transcript of the interaction in the first phase, there exists at most one {\it value\/} that the receiver can later (i.e., in the second phase) accept as a legal ``opening'' of the commitment.
\end{enumerate}
In addition, the protocol should be {\it viable\/}, in the sense that at the end of the second phase, the receiver should receive the {\it value\/} that the sender committed to. The first phase is called the ``commit phase'' while the second phase is called the ``reveal phase''. Formally, we have the following definition.
\begin{definition}{Bit-Commitment Scheme} A {\bf bit-commitment scheme} is a pair of {\it probabilistic polynomial-time interactive machines\/}, denoted $(S,R)$, satisfying the following:
\begin{itemize}
\item Input: $n$ as security parameter.
\item {\it Hiding\/}: The commitment of $0$ is indistinguishable from the commitment of $1$, namely:
$$
    \{\langle S(0),R^*\rangle(1^n)\}\cequiv\{\langle S(1),R^*\rangle(1^n)\}
$$
\item {\it Binding}: Some preliminaries to the requirement:
    \begin{enumerate}
        \item Let $(r,\bar{m})$ be the view of the receiver. we say that $(r,\bar{m})$ is a possible $\sigma$-commitment if $(r,\bar{m})=\view_{R(1^n,r)}^{S(\sigma,1^n,s)}$
        \item {\it Ambiguous}: We say the receiver's view $(r,\bar{m})$ is {\bf ambiguous} if it is both a possible 0-commmitment and a possible 1-commitment.
    \end{enumerate}
The {\it binding propertiy} requires that for all but a negligible fraction of $r\in\{0,1\}^{poly(n)}$ there is no $\bar{m}$ such that $(r,\bar{m})$ is ambiguous.
\end{itemize}
\end{definition}
\begin{description}
\item[Canonical Reveal Phase.] This can be done easily by:
\begin{enumerate}
    \item The sender sends to the receiver its initial private input $v$ and the random coins $s$ it used in the commit phase.
    \item The receiver verifies that $v$ and $s$ indeed yeild the messages that $R$ received in the commit phase.
\end{enumerate}
\end{description}

\subsection{Construction Based on Any One-Way Permutation}