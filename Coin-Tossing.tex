\section{Determining Output - Coin Tossing}

In this section, we present the protocol by Blum for tossing a single coin securely. The protocol securely computes the functionality $f_{ct}(\lambda,\lambda)=(U_1,U_1)$.
Intuition behind protocol~\ref{proto:ct-1bit} is simple: each party locally choose a random bit, and the result is the XOR of the two bits. But there is a problem that if $P_1$ sends its random bit to $P_2$ first, $P_2$ is able to cheat and force the output to be the result it desires. This can be fixed by requiring $P_2$ commit its random bit before $P_1$ sending the random bit.

\begin{adjustbox}{minipage=1\linewidth,fbox}
\begin{protocol}[Blum's Coin Tossing of a Single Bit]$ $
    \begin{description}
        \item[Security parameter:] Both parties have security parameter $1^n$
        \item[The proof system:]
        $ $\newline\newline
        \begin{adjustbox}{minipage=0.6\linewidth}
            \begin{tabular}{lcl}
            $P_1(1^n)$ & & $P_2(1^n)$ \\
            \\
            Choose $b_1\in_R\{0,1\}$ and \\
            a random $r\in\{0,1\}^n$ and send \\
            $c=\com(b_1;r)$ & \rextlinearrow{c}{15} & \\
            &\lextlinearrow{b_2}{15} & $b_2\in_R\{0,1\}$ \\
            {\bf If} $b_2$ is invalid, let $b_2=0$ \\
            {\bf Output} $b=b_1\oplus b_2$ & \rextlinearrow{(b_1,r)}{15} & \\
            & & checks that $c=\com(b_1;r)$ \\
            & & {\bf If} yes, output $b=b_1\oplus b_2$\\
            & & {\bf Otherwise} output $\perp$
            \end{tabular}
        \end{adjustbox}
        \\
    \end{description}
    \label{proto:ct-1bit}
\end{protocol}
\end{adjustbox}
\\ \\
In the simulation, the simulator receives the output bit $b$ from the trusted party and needs to make the result of the execution equal $b$.

\begin{algorithm}
\SetKwInOut{Input}{Input}
\SetKwInOut{Output}{Output}
$\simu^{\calA(\cdot)}()$\;
\Input{Nothing}
\Output{A simulated view and output close to real view of $\calA$.}
``Send'' $\lambda$ to $f_{ct}$ and receives bit $b$\;
\For{$i=1,\ldots,n$}{
    $b_1\stackrel{\$}{\gets}\{0,1\}$\;
    $r\stackrel{\$}{\gets}\{0,1\}^n$\;
    $c=\com(b_1;r)$\;
    Let $b_2=\calA(c)$, note that invalid $b_2$ is interpreted as $b_2=0$\;
    \If{$b_2 = b\oplus b_1$}{
        {\bf Output} $\calA(c,(b_1,r))$
    }
}
{\bf Output} {\sf fail}.

\caption{Simulator for $P_2$ in Protocol~\ref{proto:ct-1bit}}\label{alg:ct-1bit-sim}
\end{algorithm}

\begin{theorem} Assume that $\com$ is a {\it perfectly-binding} commitment scheme. Then, Protocol \ref{proto:ct-1bit} securely computes the bit coin-tossing functionality defined by $f_{ct}(\lambda, \lambda)=(U_1,U_1).$
\end{theorem}
\begin{proof}
First, consider the case where $P_2$ is corrupted. The simulator is defined as Algorithm \ref{alg:bot-p2-sim}. We first prove that $\simu$ outputs {\sf fails} with negligible probability.
Observe that the iterations finish if $\calA(\com(b_1))=b\oplus b_1$. We have:
    \begin{align*}
        \Pr[\calA(\com(b_1))=b_1\oplus b]&=\frac{1}{2}\Pr[\calA(\com(0))=b]+\frac{1}{2}\Pr[\calA(\com(1))=1\oplus b]\\
        &=\frac{1}{2}\Pr[\calA(\com(0))=b]+\frac{1}{2}(1-\Pr[\calA(\com(0))=b]) \\
        &=\frac{1}{2}+\frac{1}{2}(\Pr[\calA(\com(0))=b]-\Pr[\calA(\com(1))=b])
    \end{align*}
by the computationally hiding property of $\com$ we have that exists a negligible function $\mu(\cdot)$ such that:
$$
    \abs{\Pr[\calA(\com(0))=b]-\Pr[\calA(\com(1))=b]}<\mu(n)
$$
and so
\begin{equation}
    \frac{1}{2}\cdot(1-\mu(n))\leq\Pr[\calA(\com(b_1))=b_1\oplus b]\leq\frac{1}{2}(1+\mu(n))\label{eq:com-hiding}
\end{equation}
The simulator fails with probability as most
$$
    \left(\frac{1}{2}\cdot(1+\mu(n))\right)^n <\left(\frac{2}{3}\right)^n
$$
Next, we show that conditioned on that $\simu$ doesn't output {\sf fails}, the output distribution of $\ideal$ and $\real$ are statistically close. We know that both distribution is of form $(b_1, \calA(\com(b_1;r), b_1, r))$. The difference is as follows:
    \begin{itemize}
        \item In $\real$, $b_1, r$ is sampled uniformly.
        \item In $\ideal$, a random $b$ is chosen, and thtn random $b_1$ and $r$ are chosen under the constraint that $b_1\oplus\calA(\com(b_1;r))=b$.
    \end{itemize}
In $\real$ world, $(\hat{b_1},\hat{r})$ appears with probability exactly $2^{-(n+1)}$. Regarding the $\ideal$ world, let $S_b=\{(b_1,r)|b_1\oplus\calA(\com(b_1;r))=b\}$. Any fixed $(\hat{b_1},\hat{r})$ appears with probability
$$
    \frac{1}{2}\cdot\frac{1}{\abs{S_b}}.
$$
and $\abs{S_b}$ can be bounded by equation \ref{eq:com-hiding}. We conclude that $\ideal$ and $\real$ is statistically close.

Next, consider $P_1$ is corrupted.
\end{proof}

\begin{algorithm}
\SetKwInOut{Input}{Input}
\SetKwInOut{Output}{Output}
$\simu^{\calA(\cdot)}()$\;
\Input{Nothing}
\Output{A simulated view and output close to real view of $\calA$.}
{\bf Send} $\lambda$ to $f_{ct}$ and receives bit $b$\;
{\bf Recv} $c$ from $\calA$\;
{\bf Send} $b_2=0$ to $\calA$ and receives $({b'}_1^0,{r'}_1^0)$.\;
{\bf Send} $b_2=1$ to $\calA$ and receives $({b'}_1^1,{r'}_1^1)$.\;

\caption{Simulator for $P_1$ in Protocol~\ref{proto:ct-1bit}}\label{alg:ct-1bit-sim-p2}
\end{algorithm}